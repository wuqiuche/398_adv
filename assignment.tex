\documentclass[a4paper,10pt,nil]{article}
\usepackage{setspace}
\usepackage{enumerate}
% Additional Options
\begin{document}
 \begin{spacing}{1.5}

\textbf{Q1:} Suppose f(x) is continuous on the interval [0,1], and $f(0)=f(1)$, try to prove:
\begin{enumerate}[(1)]
\item there exists $x \in$ [0,1], such that $f(x)=f(x+\frac{1}{2})$;
\item for $\forall n\in Z^+$, there exists $x \in$ [0,1], such that $f(x)=f(x+\frac{1}{n})$
\end{enumerate}

\textbf{Solution:}
\begin{enumerate}[(1)] 
\item Let $g(x)=f(x)-f(x+\frac{1}{2})$,\\ then we have $g(0)=f(0)-f(\frac{1}{2}), g(\frac{1}{2})=f(\frac{1}{2})-f(1)$.\\ Then, $g(0)+g(\frac{1}{2})=f(0)-f(1)=0$\\
So, due to the intermediate value theorem, there must exist a number $x \in$ [0,1], such that $g(x)=\frac{g(0)+g(\frac{1}{2})}{2}=0$. \\
So, it is proved.
\item Let $h(x)=f(x)-f(x+\frac{1}{n})$,\\ then $h(0)=f(0)-f(\frac{1}{n}),h(\frac{1}{n})=f(\frac{1}{n})-f(\frac{2}{n}),...,h(\frac{n-1}{n})=f(\frac{n-1}{n})-f(1).$\\
Add these equations, we have $\sum_{i=0}^{n-1}h(\frac{i}{n})=f(0)-f(1)=0.$
Due to the intermediate value theorem, there must exist a number $x \in$ [0,1], such that $h(x)=\frac{1}{n}\sum_{i=0}^{n-1}h(\frac{i}{n})=0$ \\
So it is proved.
\end{enumerate}

\end{spacing}
\end{document}